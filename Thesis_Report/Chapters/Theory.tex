% Background Literature survey


\chapter{Theory}

Crystallization is the (natural or artificial) process where a solid forms where the atoms or molecules are highly organized in a structure known as a crystal. Some of the ways which crystals form are through precipitating from a solution, melt or more rarely deposited directly from a gas. \\
Crystal shapes can include cubic, tetragonal, orthorhombic, hexagonal, monoclinic, triclinic, and trigonal. In order for crystallization to take place a solution must be "supersaturated". Supersaturation refers to a state in which the liquid (solvent) contains more dissolved solids (solute) than can ordinarily be accommodated at that temperature.

Supersaturation , can be mathematically defined as :\\
\begin{math}
\textbf{Supersaturation} \( = \Delta{C} = C - C_{s} \) \\
\textbf{Relative Supersaturation} \( = \Delta{C} / C_{s} = S\) \\

\end{math} 

C\textsubscript{s} : Concentration of solute in saturated solution \\
C :   Concentration of solute in the solution \\
S :   Supersaturation ratio \\
\paragraph{}
The crystallization process consists of two major type of kinetics, \textit{nucleation} and \textit{crystal growth} which are driven by thermodynamic properties as well as chemical properties.

\begin{itemize}
\item \textbf{Nucleation} is the step where the solute molecules or atoms dispersed in the solvent start to gather into clusters, on the microscopic scale (elevating solute concentration in a small region). These stable clusters constitute the nuclei.

\item \textbf{Crystal growth} is the subsequent size increase of the nuclei that succeed in achieving the critical cluster size. it is a dynamic process occurring in equilibrium where solute molecules or atoms precipitate out of solution, and dissolve back into solution.

\end{enumerate}
\\

Supersaturation is the practical driving forces of a crystallization process. Depending upon the conditions, either nucleation or growth may be predominant over the other, dictating crystal size.\\
Two other phenomena which are often neglected in the crystallization modelling process are \textbf{Agglomeration} and \textbf{Breakage}. Agglomeration occurs when two particles collide and stick together to form a larger particle. Breakage occurs in stirred vessels; the larger particle breaks into smaller fragments, because of attrition .


\section{Nucleation}

This is the initial process for the formation of a crystals in a solution, a liquid, or a vapour, in which a small number of ions, atoms, or molecules become arranged in a characteristic pattern of a crystalline solid, and subsequently form a site upon which additional particles are deposited as the crystal grows.\\
Nucleation requires supersaturation, which is obtained usually by a change in temperature (cooling in case of a positive gradient of the solubility curve and heating in case of a negative gradient), by removing the solvent, or by adding a drowning out agent or reaction partners. If the solution contains neither solid foreign particles or crystals of its own type, nuclei are formed only through homogeneous nucleation. If foreign particles are present then nuclei are formed through heterogeneous nucleation.\\
Both homogeneous nucleation and heterogeneous nucleation are classified as primary nucleation.


\section{Crystal Growth}

Crystal growth occurs as soon as nuclei with radius larger than the critical radius have been formed. There are many proposed mechanisms for crystal growth.\\
Diffusion theories assume that matter is deposited continuously on the crystal face at a rate proportional to the difference in concentration between the point of deposition and the bulk of the solution.\\
When dealing with crystal growth in an ionizing solute, the following steps can be distinguished :
\begin{itemize}
\item Bulk diffusion of solvated ions through the diffusion boundary layer
\item Bulk diffusion of solvated ions through adsorption layer
\item Surface diffusion of solvated or unsolvated ions
\item Partial or total desolvation of ions
\item Integration of ions into the lattice
\item Counterdiffusion through adsorption layer of water released
\item Counterdiffusion of water through the boundary layer

\end{itemize}

\textit{The slowest of these steps are rate determining.} \\

A crystal surface grow in such a way that units in a supersaturated solution are first transported by diffusion and convection and then built into the surface of the crystal by integration or an integration reaction , with the supersaturation, $\Delta{C}$, being the driving force.\\
Thus, determination of the optimal temperature or supersaturation trajectory for a seeded batch crystallizer is the most well-studied problem in chemical engineering, apart from batch reactors and batch distillation as the evolution of supersaturation in time affects almost all the kinetic phenomena occurring in the crystallization process. For example, growth can be size-independent or size-dependent; it can have a constant value or it may be a function of a thermodynamic parameter such as solubility and thus selection of appropriate kinetics is essential for accurate modelling .

