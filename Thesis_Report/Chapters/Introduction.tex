\chapter{Introduction}
Batch crystallization is widely used in chemical, pharmaceutical, photographic, and other manufacturing processes for the preparation of crystalline products with several desirable attributes. \\
The batch system helps to obtain a narrower Particle size distribution (PSD) with high crystal purity. The crystallization process has an influence on the downstream processing and, hence, reproducible PSD in each operation is of prime importance. Thus, it is essential to find the variables affecting the process and control them within an acceptable range, so as to satisfy the final product quality requirements.
Considering the operation of crystallizers, a batch process is preferable as a larger mean crystal size and narrower Crystal size distribution (CSD) can be achieved. In general, the CSD which is typically characterized by the mean and variance of crystal size is a key property to control this process because it directly affects final product qualities. Therefore, finding effective control strategy to obtain the crystals with a desired CSD is significant in order for improving the performance of batch crystallization processes and at the same time reducing difficulties in downstream processing.
In the following work we formulate the problem using the population balance equations and obtain the solution for the optimal Temperature profile using Deterministic and Probabilistic methods. 

\paragraph{Deterministic Optimal Control} aims at finding the an optimum temperature profile to maximise an objective function selected to achieve a desired volume of the product. Herein, the experimental kinetic parameters are employed to simulate a batch crystalllization process.

\paragraph{Stochastic Optimal Control} undertakes the task of quantifying the uncertainites which creep in due to experimentation. It aims to achive a maximum expected value for the desired product, simultaneously incorporating  randomness in the process parameters into the model. Namely, 2 methods \textbf{Ito Process} and \textbf{Polynomial Chaos Expansions} are employed for this purpose.


\section{Previous Work}

The concept of programmed cooling in batch crystallizers was first discussed by Mullin and Nyvlt \cite{mullin} in 1971. They studied the laboratory-scale crystallization of potassium sulfate and ammonium sulfate using a temperature controller and observed improvement in the crystal size and quality under programmed cooling. \\
Later, in 1974, A. G. Jones \cite{agjones} presented a mathematical theory based on moment transformations of population balance equations. He used the continuous maximum principle to predict optimal cooling curves.
Rawlings et al. \cite{rawlings} discussed issues in crystal size measurement using laser light scattering experiments and optimal control problem formulation. In 1994, Miller and Rawlings \cite{miller_rawlings}  discussed the uncertain bounds on model parameter estimates for a batch crystallization system.\\ 
Most importantly optimal temperature prediction for batch crystallization has also been done by Hu et al.\cite{hu}, Shi et al.\cite{shi}, Paengjuntuek et al.\cite{paeng}, and Corriou and Rohani.\cite{corriou}, the data and knowledge from which have been used in further work in this project . Grosso et al.\cite{grosso} presented a stochastic approach for modeling PSD and comparative assessments of different models. Ma et al.\cite{ma} presented a worse-case performance analysis of optimal control trajectories by considering features such as the computational effort, parametric uncertainty and control implementation inaccuracies. \\
The focus of the current work is to be able to handle parametric uncertainties in mathematical formulations of batch crystallization process.
