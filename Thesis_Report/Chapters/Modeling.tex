% Modeling a batch crystallizer

\chapter{Modelling a Seeded Batch Crystallizer}


\section{Population Balance Equation}

Analysis of a particulate system seeks to synthesize the behavior of the population of particles and its environment from the behavior of single particles in their local environments. The population is described by the density of a suitable extensive variable, usually the \textbf{number of particles}, but sometimes by other variables such as the mass or volume of particles. The usual transport equations expressing conservation laws for material systems apply to the behavior of single particles. Particulate processes are characterized by properties such as particle shape, size, surface area, mass, and product purity. \\
A population balance formulation describes the process of crystal size distribution with time most effectively. Thus, modeling of a batch crystallizer involves the use of population balances to model the crystal size prediction and the mass balance on the system can be modeled as a simple differential equation having concentration as the state variable.
The population balance can be expressed as eq :

\begin{equation}
	\frac{\partial{n(r,t)}}{\partial{t}} + \frac{\partial{G(r,t)n(r,t)}}{\partial{r}} = B

\end{equation}
where \textbf{n} is the number density distribution, \textbf{t} is the time, \textbf{r} represents the characteristic dimension for size measurements, \textbf{G} is the crystal growth rate, and \textbf{B} is the nucleation rate. Both growth and nucleation processes describe crystallization kinetics, and their expression may vary, depending on the system under consideration.

\subsection{The One-Dimensional Case}


Consider a population of particles distributed according to their size x which we shall take to be the mass of the particle and allow it to vary between 0 and infinity.Such a situation can be approximated, for example, in a crystallizer containing a highly supersaturated solution of the crystallizing solute. The process involves nucleation resulting in the formation of a rudimentary particle and its subsequent growth by transferring solute from the solution phase to the particle surface.\\
If the supersaturation is sufficiently high, the nucleation and growth rates may remain relatively unaffected as crystallization progresses. This unnecessarily restrictive assumption is made only for simplifying the preliminary derivation of the population balance equation.\\
We let X(x,t) be the growth rate of the particle of size x. The particles may then be viewed as distributed along the size coordinate and embedded on a string deforming with velocity X(x,t). Choose an arbitrary region [a, b] on the stationary size coordinate with respect to which the string with the embedded particles is deforming. 
As the string deforms, particles commute through the interval [a, b] across the endpoints a and b, changing the number of particles in the interval.If we denote the number density by f1(x,t), the rate of change in the number of particles in [a, b] caused by this traffic at a and b is given by
