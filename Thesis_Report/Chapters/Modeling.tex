% Modeling a batch crystallizer

\chapter{Modelling a Seeded Batch Crystallizer}


\section{Population Balance Equation}

Analysis of a particulate system seeks to synthesize the behavior of the population of particles and its environment from the behavior of single particles in their local environments. The population is described by the density of a suitable extensive variable, usually the \textbf{number of particles}, but sometimes by other variables such as the mass or volume of particles. The usual transport equations expressing conservation laws for material systems apply to the behavior of single particles. Particulate processes are characterized by properties such as particle shape, size, surface area, mass, and product purity. \\
A population balance formulation describes the process of crystal size distribution with time most effectively. Thus, modeling of a batch crystallizer involves the use of population balances to model the crystal size prediction and the mass balance on the system can be modeled as a simple differential equation having concentration as the state variable.
The population balance can be expressed as eq :

\begin{equation}
	\frac{\partial{n(r,t)}}{\partial{t}} + \frac{\partial{G(r,t)n(r,t)}}{\partial{r}} = B

\end{equation}
where \textbf{n} is the number density distribution, \textbf{t} is the time, \textbf{r} represents the characteristic dimension for size measurements, \textbf{G} is the crystal growth rate, and \textbf{B} is the nucleation rate. Both growth and nucleation processes describe crystallization kinetics, and their expression may vary, depending on the system under consideration.

\section{Model Equations}

In this work, the system under consideration is potassium sulfate, which has been studied earlier by Hu et al. \cite{hu}, Shi et al. \cite{shi}, and Paengjuntuek et al.\cite{paeng}. \\

Nucleation kinetics$^{(5-7)}$ are defined by :
\begin{equation}
B(t) = k_{b}\exp{\left(-E_{b}/RT \right)}\left(\frac{C - C_{s}(T)}{C_{s}(T)}\right)^{b}\mu_{3}
\end{equation}  


Growth Kinetics$^{(5-7)}$ are given by:
\begin{equation}
G(t) = k_{g}\exp{\left(-E_{g}/RT \right)}\left(\frac{C - C_{s}(T)}{C_{s}(T)}\right)^{g}
\end{equation}
where k$_{b}$ and k$_{g}$ are constants of the system, E$_{b}$ and E$_{g}$ are activation energies, and b and g are exponents of nucleation and growth, respectively. $C_{s}(T)$ is the saturation concentration at a given temperature. The following equations are used to evaluate the saturation and metastable concentrations corresponding to the solution temperature T (expressed in units of $^\circ$C)\cite{shi}.
\begin{align}
C_{s}(T) &= 6.29\times10^{-2} + 2.46\times10^{-3}T - 7.14\times10^{-6}T^{2} \\
C_{m}(T) &= 7.76\times10^{-2} + 2.46\times10^{-3}T - 8.1\times10^{-6}T^{2}
\end{align} 
The mass balance, in terms of concentration of the solute in the solution, is expressed as :
\begin{equation}
\frac{dC}{dt} = -3\rho{}k_{v}G(t)\mu_{2}(t)
\end{equation}
where $\rho{}$ is the density of the crystals, $k_{v}$ the volumetric shape factor, and $\mu_{2}$ is the second moment of particle size distribution (PSD).\\

Since $n(r,t)$ represents the population density of the crystals\cite{randolph}, the ith moment of the particle size distribution(PSD) is given by :
\begin{equation}
\mu_{i} = \int_{0}^{\infty} r^{i}n(r,t) dr
\end{equation}

The above equations along with the Population Balance Equation represent a complete model of a seeded batch crystallizer . 
Since population balance equations are multidimensional, their implementation in control functions is tedious; hence, much research has been focused on the model order reduction methods.\\
For simplifying the solution method, we reduce the population balance equations into moment balance equations. It is also advantageous, since it is difficult and time-consuming to formulate an optimization problem involving PBEs. Thus, the moment model leads to a reduced-order model involving the process dynamics in batch crystallization.

%%%% Insert table of parameters

\section{Solution Methodology}

Separate moment equations are used for the seed and nuclei classes of crystals, and they are defined as : 
\begin{align}
\mu^{n}_{i} = \int_{0}^{r_{g}} r^{i}n(r,t) dr \\
\mu^{s}_{i} = \int_{r_{g}}^{\infty} r^{i}n(r,t) dr
\end{align}
n : nucleation , s: seed , $r_{g}$  : critical radius separating the two \\  
Since, we ignore the agglomeration and breakage phenomena, the number of seeds added to the process ($\mu_{0}^{s}$) remain constant.

The differential equations for nucleated and seeded crystals are as follows :

\begin{enumerate}

\item Nucleated crystals$^{5,7}$ 
\begin{align}
\frac{du_{0}^{n}}{dt} &= B(t) \\
\frac{du_{i}^{n}}{dt} &= iG(t)u_{i-1}^{n}(t) \quad  i = 1,2,3
\end{align}

\item Seeded crystals$^{5,7}$
\begin{equation}
\frac{du_{i}^{s}}{dt} = iG(t)u_{i-1}^{n}(t) \quad  i = 1,2,3 \\
\end{equation}

\end{enumerate}
The total moment is obtained as the summation $\mu_{i}^{t} = \mu_{i}^{n} + \mu_{i}^{s}$