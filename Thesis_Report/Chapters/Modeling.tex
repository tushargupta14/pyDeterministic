% Modeling a batch crystallizer

\chapter{Modelling a Seeded Batch Crystallizer}


\section{Population Balance Equation}

Analysis of a particulate system seeks to synthesize the behavior of the population of particles and its environment from the behavior of single particles in their local environments. The population is described by the density of a suitable extensive variable, usually the \textbf{number of particles}, but sometimes by other variables such as the mass or volume of particles. The usual transport equations expressing conservation laws for material systems apply to the behavior of single particles. Particulate processes are characterized by properties such as particle shape, size, surface area, mass, and product purity. \\
A population balance formulation describes the process of crystal size distribution with time most effectively. Thus, modeling of a batch crystallizer involves the use of population balances to model the crystal size prediction and the mass balance on the system can be modeled as a simple differential equation having concentration as the state variable.
The population balance can be expressed as eq :

\begin{equation}
	\frac{\partial{n(r,t)}}{\partial{t}} + \frac{\partial{G(r,t)n(r,t)}}{\partial{r}} = B

\end{equation}
where \textbf{n} is the number density distribution, \textbf{t} is the time, \textbf{r} represents the characteristic dimension for size measurements, \textbf{G} is the crystal growth rate, and \textbf{B} is the nucleation rate. Both growth and nucleation processes describe crystallization kinetics, and their expression may vary, depending on the system under consideration.

\section{Model Equations}

In this work, the system under consideration is potassium sulfate, which has been studied earlier by Hu et al. \cite{hu}, Shi et al. \cite{shi}, and Paengjuntuek et al.\cite{paeng}. \\

Nucleation kinetics$^{(5-7)}$ are defined by :

\begin{equation*}

B(t) = k_{b}\exp{\left(-E_{b}/RT \right)}\left(\frac{C - C_{s}(T)}{C_{s}(T)}\right)^{b}\mu_{3}

\end{equation*}    
