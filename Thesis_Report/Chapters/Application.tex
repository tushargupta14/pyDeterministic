\chapter{Application}

\paragraph{Aim} To build a predictive model for unseeded batch crystallization of L-Asparagine Monohyrdate(LAM). The problem here follows only Nucleation kinectics unlike the previous versions and thus proves an ideal opportunity to diversify in application.


\section{Optimal Control Problem}

The population balance equations(PBE) given by \ref{populationbalance} are used here to model the kinetics of crystal formation. The Nucleation rate expression for LAM crystals is given by\cite{lindenberg} :
\begin{equation}
B = k_{j_{1}}S\exp\left( -k_{j_{2}}\frac{\ln^{3}{C_{c}/C^{*}}}{\ln^{2}S}\right) 
\end{equation}
$C_{c}$ representys the  molar density of LAM. kJ1 and kJ2 are
empirical parameters. The following power-law expression is used to describe the growth rate\cite{nagy}\cite{nagy2}:
\begin{equation}
G = k_{g}(S-1)^{g}
\end{equation}
The supersaturation ratio, S, is defined as :
\begin{equation}
S = C/C^{*}
\end{equation}
C* represents the saturation concentration of LAM. The solubility of LAM in 
can be expressed as \cite{}:
\begin{equation}
C^{*} = 5 \times 10^{-5}T^{2} - 0.001T + 0.0236
\end{equation}
Method of Meoments has been used to reduce the PBE to ODE's as stated in Section \ref{modeleq}. The mass balance for LAM crystals also remains the same from there, with the difference being the absence of seeded crystals.\\

\section{Solution Technique}
The ODEs for the model are given by :

\begin{align}
\frac{du_{0}}{dt} = B
\frac{du_{j}}{dt} = jG\mu_{j-1}
\end{align}
for  $j = 1,2,3,4 $
\\
The determination of the optimal temperature or supersaturation profile for maximizing the weight mean size is a highly studied objective for a crystallization process.
\paragraph{Objective Function} becomes :
