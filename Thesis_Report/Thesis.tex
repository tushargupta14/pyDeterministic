
%% ----------------------------------------------------------------
%% Thesis.tex -- MAIN FILE (the one that you compile with LaTeX)
%% ---------------------------------------------------------------- 

% Set up the document
\documentclass[a4paper, 11pt, oneside]{Thesis}  % Use the "Thesis" style, based on the ECS Thesis style by Steve Gunn
\usepackage{graphicx}
\graphicspath{images/}
%\graphicspath{Figures/}  % Location of the graphics files (set up for graphics to be in PDF format)

% Include any extra LaTeX packages required

\usepackage[square, numbers, comma, sort&compress]{natbib}  % Use the "Natbib" style for the references in the Bibliography
\usepackage{verbatim}  % Needed for the "comment" environment to make LaTeX comments
\usepackage{vector}  % Allows "\bvec{}" and "\buvec{}" for "blackboard" style bold vectors in maths
%\hypersetup{urlcolor=blue, colorlinks=true}  % Colours hyperlinks in blue, but this can be distracting if there are many links.

%% ----------------------------------------------------------------
\begin{document}
\frontmatter      % Begin Roman style (i, ii, iii, iv...) page numbering

% Set up the Title Page
\title  {Optimal Control of Seeded Batch Crystallizer}
\authors  {\texorpdfstring
            {\href{your web site or email address}{Author Name}}
            {Tushar Gupta}
            }
\addresses  {\groupname\\\deptname\\\univname}  % Do not change this here, instead these must be set in the "Thesis.cls" file, please look through it instead
\date       {\today}
\subject    {}
\keywords   {}

\maketitle
%% ----------------------------------------------------------------

\setstretch{1.3}  % It is better to have smaller font and larger line spacing than the other way round

% Define the page headers using the FancyHdr package and set up for one-sided printing
\fancyhead{}  % Clears all page headers and footers
\rhead{\thepage}  % Sets the right side header to show the page number
\lhead{}  % Clears the left side page header

\pagestyle{fancy}  % Finally, use the "fancy" page style to implement the FancyHdr headers

%% ----------------------------------------------------------------
% Declaration Page required for the Thesis, your institution may give you a different text to place here
\Declaration{

\addtocontents{toc}{\vspace{1em}}  % Add a gap in the Contents, for aesthetics

I, AUTHOR NAME, declare that this thesis titled, `THESIS TITLE' and the work presented in it are my own. I confirm that:

\begin{itemize} 
\item[\tiny{$\blacksquare$}] This work was done wholly or mainly while in candidature for a research degree at this University.
 
\item[\tiny{$\blacksquare$}] Where any part of this thesis has previously been submitted for a degree or any other qualification at this University or any other institution, this has been clearly stated.
 
\item[\tiny{$\blacksquare$}] Where I have consulted the published work of others, this is always clearly attributed.
 
\item[\tiny{$\blacksquare$}] Where I have quoted from the work of others, the source is always given. With the exception of such quotations, this thesis is entirely my own work.
 
\item[\tiny{$\blacksquare$}] I have acknowledged all main sources of help.
 
\item[\tiny{$\blacksquare$}] Where the thesis is based on work done by myself jointly with others, I have made clear exactly what was done by others and what I have contributed myself.
\\
\end{itemize}
 
 
Signed:\\
\rule[1em]{25em}{0.5pt}  % This prints a line for the signature
 
Date:\\
\rule[1em]{25em}{0.5pt}  % This prints a line to write the date
}
\clearpage  % Declaration ended, now start a new page

%% ----------------------------------------------------------------
% The "Funny Quote Page"
\pagestyle{empty}  % No headers or footers for the following pages

\null\vfill
% Now comes the "Funny Quote", written in italics
\textit{``Write a funny quote here.''}

\begin{flushright}
If the quote is taken from someone, their name goes here
\end{flushright}

\vfill\vfill\vfill\vfill\vfill\vfill\null
\clearpage  % Funny Quote page ended, start a new page
%% ----------------------------------------------------------------

% The Abstract Page
\addtotoc{Abstract}  % Add the "Abstract" page entry to the Contents
\abstract{
\addtocontents{toc}{\vspace{1em}}  % Add a gap in the Contents, for aesthetics

The Thesis Abstract is written here (and usually kept to just this page). The page is kept centered vertically so can expand into the blank space above the title too\ldots

}

\clearpage  % Abstract ended, start a new page
%% ----------------------------------------------------------------

\setstretch{1.3}  % Reset the line-spacing to 1.3 for body text (if it has changed)

% The Acknowledgements page, for thanking everyone
\acknowledgements{
\addtocontents{toc}{\vspace{1em}}  % Add a gap in the Contents, for aesthetics

The acknowledgements and the people to thank go here, don't forget to include your project advisor\ldots

}
\clearpage  % End of the Acknowledgements
%% ----------------------------------------------------------------

\pagestyle{fancy}  %The page style headers have been "empty" all this time, now use the "fancy" headers as defined before to bring them back


%% ----------------------------------------------------------------
\lhead{\emph{Contents}}  % Set the left side page header to "Contents"
\tableofcontents  % Write out the Table of Contents

%% ----------------------------------------------------------------
\lhead{\emph{List of Figures}}  % Set the left side page header to "List if Figures"
\listoffigures  % Write out the List of Figures

%% ----------------------------------------------------------------
\lhead{\emph{List of Tables}}  % Set the left side page header to "List of Tables"
\listoftables  % Write out the List of Tables

%% ----------------------------------------------------------------
\setstretch{1.5}  % Set the line spacing to 1.5, this makes the following tables easier to read
\clearpage  % Start a new page
\lhead{\emph{Abbreviations}}  % Set the left side page header to "Abbreviations"
\listofsymbols{ll}  % Include a list of Abbreviations (a table of two columns)
{
% \textbf{Acronym} & \textbf{W}hat (it) \textbf{S}tands \textbf{F}or \\
\textbf{LAH} & \textbf{L}ist \textbf{A}bbreviations \textbf{H}ere \\

}

%% ----------------------------------------------------------------
\clearpage  % Start a new page
%\lhead{\emph{Physical Constants}}  % Set the left side page header to "Physical Constants"
\listofconstants{lrcl}  % Include a list of Physical Constants (a four column table)
{
% Constant Name & Symbol & = & Constant Value (with units) \\
Speed of Light & $c$ & $=$ & $2.997\ 924\ 58\times10^{8}\ \mbox{ms}^{-\mbox{s}}$ (exact)\\

}

%% ----------------------------------------------------------------
\clearpage  %Start a new page
%\lhead{\emph{Symbols}}  % Set the left side page header to "Symbols"
\listofnomenclature{lll}  % Include a list of Symbols (a three column table)
{
% symbol & name & unit \\
$a$ & distance & m \\
$P$ & power & W (Js$^{-1}$) \\
& & \\ % Gap to separate the Roman symbols from the Greek
$\omega$ & angular frequency & rads$^{-1}$ \\
}
%% ----------------------------------------------------------------
% End of the pre-able, contents and lists of things
% Begin the Dedication page

\setstretch{1.3}  % Return the line spacing back to 1.3

\pagestyle{empty}  % Page style needs to be empty for this page
\dedicatory{For/Dedicated to/To my\ldots}

\addtocontents{toc}{\vspace{2em}}  % Add a gap in the Contents, for aesthetics


%% ----------------------------------------------------------------
\mainmatter	  % Begin normal, numeric (1,2,3...) page numbering
\pagestyle{fancy}  % Return the page headers back to the "fancy" style

% Include the chapters of the thesis, as separate files\\

% Just uncomment the lines as you write the chapters

\chapter{Introduction}
Batch crystallization is widely used in chemical, pharmaceutical, photographic, and other manufacturing processes for the preparation of crystalline products with several desirable attributes. \\
The batch system helps to obtain a narrower Particle size distribution (PSD) with high crystal purity. The crystallization process has an influence on the downstream processing and, hence, reproducible PSD in each operation is of prime importance. Thus, it is essential to find the variables affecting the process and control them within an acceptable range, so as to satisfy the final product quality requirements.
Considering the operation of crystallizers, a batch process is preferable as a larger mean crystal size and narrower Crystal size distribution (CSD) can be achieved. In general, the CSD which is typically characterized by the mean and variance of crystal size is a key property to control this process because it directly affects final product qualities. Therefore, finding effective control strategy to obtain the crystals with a desired CSD is significant in order for improving the performance of batch crystallization processes and at the same time reducing difficulties in downstream processing.
In the following work we formulate the problem using the population balance equations and obtain the solution for the optimal Temperature profile using Deterministic and Probabilistic methods 


\paragraph{Deterministic Optimal Control} 

Quisque tristique urna in lorem laoreet at laoreet quam congue. Donec dolor turpis, blandit non imperdiet aliquet, blandit et felis. In lorem nisi, pretium sit amet vestibulum sed, tempus et sem. Proin non ante turpis. Nulla imperdiet fringilla convallis. Vivamus vel bibendum nisl. Pellentesque justo lectus, molestie vel luctus sed, lobortis in libero. Nulla facilisi. Aliquam erat volutpat. Suspendisse vitae nunc nunc. Sed aliquet est suscipit sapien rhoncus non adipiscing nibh consequat. Aliquam metus urna, faucibus eu vulputate non, luctus eu justo.

\paragraph{Stochastic Optimal Control}

Donec urna leo, vulputate vitae porta eu, vehicula blandit libero. Phasellus eget massa et leo condimentum mollis. Nullam molestie, justo at pellentesque vulputate, sapien velit ornare diam, nec gravida lacus augue non diam. Integer mattis lacus id libero ultrices sit amet mollis neque molestie. Integer ut leo eget mi volutpat congue. Vivamus sodales, turpis id venenatis placerat, tellus purus adipiscing magna, eu aliquam nibh dolor id nibh. Pellentesque habitant morbi tristique senectus et netus et malesuada fames ac turpis egestas. Sed cursus convallis quam nec vehicula. Sed vulputate neque eget odio fringilla ac sodales urna feugiat.
\textbf{Ito Process} and \textbf{Polynomial Chaos Expansions}
 % Introduction

% Background Literature survey


\chapter{Theory}

Crystallization is the (natural or artificial) process where a solid forms where the atoms or molecules are highly organized in a structure known as a crystal. Some of the ways which crystals form are through precipitating from a solution, melt or more rarely deposited directly from a gas. \\
Crystal shapes can include cubic, tetragonal, orthorhombic, hexagonal, monoclinic, triclinic, and trigonal. In order for crystallization to take place a solution must be "supersaturated". Supersaturation refers to a state in which the liquid (solvent) contains more dissolved solids (solute) than can ordinarily be accommodated at that temperature.

Supersaturation , can be mathematically defined as :\\
\begin{math}
\textbf{Supersaturation} \( = \Delta{C} = C - C_{s} \) \\
\textbf{Relative Supersaturation} \( = \Delta{C} / C_{s} = S\) \\

\end{math} 

C\textsubscript{s} : Concentration of solute in saturated solution \\
C :   Concentration of solute in the solution \\
S :   Supersaturation ratio \\
\paragraph{}
The crystallization process consists of two major type of kinetics, \textit{nucleation} and \textit{crystal growth} which are driven by thermodynamic properties as well as chemical properties.

\begin{itemize}
\item \textbf{Nucleation} is the step where the solute molecules or atoms dispersed in the solvent start to gather into clusters, on the microscopic scale (elevating solute concentration in a small region). These stable clusters constitute the nuclei.

\item \textbf{Crystal growth} is the subsequent size increase of the nuclei that succeed in achieving the critical cluster size. it is a dynamic process occurring in equilibrium where solute molecules or atoms precipitate out of solution, and dissolve back into solution.

\end{enumerate}
\\

Supersaturation is the practical driving forces of a crystallization process. Depending upon the conditions, either nucleation or growth may be predominant over the other, dictating crystal size.\\
Two other phenomena which are often neglected in the crystallization modelling process are \textbf{Agglomeration} and \textbf{Breakage}. Agglomeration occurs when two particles collide and stick together to form a larger particle. Breakage occurs in stirred vessels; the larger particle breaks into smaller fragments, because of attrition .


\section{Nucleation}

This is the initial process for the formation of a crystals in a solution, a liquid, or a vapour, in which a small number of ions, atoms, or molecules become arranged in a characteristic pattern of a crystalline solid, and subsequently form a site upon which additional particles are deposited as the crystal grows.\\
Nucleation requires supersaturation, which is obtained usually by a change in temperature (cooling in case of a positive gradient of the solubility curve and heating in case of a negative gradient), by removing the solvent, or by adding a drowning out agent or reaction partners. If the solution contains neither solid foreign particles or crystals of its own type, nuclei are formed only through homogeneous nucleation. If foreign particles are present then nuclei are formed through heterogeneous nucleation.\\
Both homogeneous nucleation and heterogeneous nucleation are classified as primary nucleation.


\section{Crystal Growth}

Crystal growth occurs as soon as nuclei with radius larger than the critical radius have been formed. There are many proposed mechanisms for crystal growth.\\
Diffusion theories assume that matter is deposited continuously on the crystal face at a rate proportional to the difference in concentration between the point of deposition and the bulk of the solution.\\
When dealing with crystal growth in an ionizing solute, the following steps can be distinguished :
\begin{itemize}
\item Bulk diffusion of solvated ions through the diffusion boundary layer
\item Bulk diffusion of solvated ions through adsorption layer
\item Surface diffusion of solvated or unsolvated ions
\item Partial or total desolvation of ions
\item Integration of ions into the lattice
\item Counterdiffusion through adsorption layer of water released
\item Counterdiffusion of water through the boundary layer

\end{itemize}

\textit{The slowest of these steps are rate determining.} \\

A crystal surface grow in such a way that units in a supersaturated solution are first transported by diffusion and convection and then built into the surface of the crystal by integration or an integration reaction , with the supersaturation, $\Delta{C}$, being the driving force.\\
Thus, determination of the optimal temperature or supersaturation trajectory for a seeded batch crystallizer is the most well-studied problem in chemical engineering, apart from batch reactors and batch distillation as the evolution of supersaturation in time affects almost all the kinetic phenomena occurring in the crystallization process. For example, growth can be size-independent or size-dependent; it can have a constant value or it may be a function of a thermodynamic parameter such as solubility and thus selection of appropriate kinetics is essential for accurate modelling .

 % Theory

% Modeling a batch crystallizer

\chapter{Modelling of a Seeded Batch Crystallizer}


\section{Population Balance Equation}

Analysis of a particulate system seeks to synthesize the behavior of the population of particles and its environment from the behavior of single particles in their local environments. The population is described by the density of a suitable extensive variable, usually the \textbf{number of particles}, but sometimes by other variables such as the mass or volume of particles. The usual transport equations expressing conservation laws for material systems apply to the behavior of single particles. Particulate processes are characterized by properties such as particle shape, size, surface area, mass, and product purity. \\
A population balance formulation describes the process of crystal size distribution with time most effectively. Thus, modeling of a batch crystallizer involves the use of population balances to model the crystal size prediction and the mass balance on the system can be modeled as a simple differential equation having concentration as the state variable.
The population balance can be expressed as eq :

\begin{equation} \label{populationbalance}
	\frac{\partial{n(r,t)}}{\partial{t}} + \frac{\partial{G(r,t)n(r,t)}}{\partial{r}} = B  \nolinebreak
\end{equation}
where \textbf{n} is the number density distribution, \textbf{t} is the time, \textbf{r} represents the characteristic dimension for size measurements, \textbf{G} is the crystal growth rate, and \textbf{B} is the nucleation rate. Both growth and nucleation processes describe crystallization kinetics, and their expression may vary, depending on the system under consideration.

\section{Model Equations} \label{modeleq}

In this work, the system under consideration is potassium sulfate, which has been studied earlier by Hu et al. \cite{hu}, Shi et al. \cite{shi}, and Paengjuntuek et al.\cite{paeng}. \\

Nucleation kinetics$^{(5-7)}$ are defined by :
\begin{equation}
B(t) = k_{b}\exp{\left(-E_{b}/RT \right)}\left(\frac{C - C_{s}(T)}{C_{s}(T)}\right)^{b}\mu_{3}
\end{equation}  


Growth Kinetics$^{(5-7)}$ are given by:
\begin{equation}
G(t) = k_{g}\exp{\left(-E_{g}/RT \right)}\left(\frac{C - C_{s}(T)}{C_{s}(T)}\right)^{g}
\end{equation}
where k$_{b}$ and k$_{g}$ are constants of the system, E$_{b}$ and E$_{g}$ are activation energies, and b and g are exponents of nucleation and growth, respectively. $C_{s}(T)$ is the saturation concentration at a given temperature. The following equations are used to evaluate the saturation and metastable concentrations corresponding to the solution temperature T (expressed in units of $^\circ$C)\cite{shi}.
\begin{align}
C_{s}(T) &= 6.29\times10^{-2} + 2.46\times10^{-3}T - 7.14\times10^{-6}T^{2} \\
C_{m}(T) &= 7.76\times10^{-2} + 2.46\times10^{-3}T - 8.1\times10^{-6}T^{2} \label{meta}
\end{align} 
The mass balance, in terms of concentration of the solute in the solution, is expressed as :
\begin{equation}
\frac{dC}{dt} = -3\rho{}k_{v}G(t)\mu_{2}(t)
\end{equation}
where $\rho{}$ is the density of the crystals, $k_{v}$ the volumetric shape factor, and $\mu_{2}$ is the second moment of particle size distribution (PSD).\\

Since $n(r,t)$ represents the population density of the crystals, the i-th moment of the particle size distribution(PSD) is given by :
\begin{equation} \label{moments}
\mu_{i} = \int_{0}^{\infty} r^{i}n(r,t) dr
\end{equation}

The above equations along with the Population Balance Equation represent a complete model of a seeded batch crystallizer . 
Since population balance equations are multidimensional, their implementation in control functions is tedious; hence, much research has been focused on the model order reduction methods.\\
For simplifying the solution method, we reduce the population balance equations into \textbf{Moment balance equations}(ODE). This is done by multiplying the equation (\ref{populationbalance})  with $r^{i}$ on both sides to generate the expression given by equation (\ref{moments}). It is also advantageous, since it is difficult and time-consuming to formulate an optimization problem involving PBEs. Thus, the moment method leads to a reduced-order model involving the process dynamics in batch crystallization.

%%%% Insert table of parameters

\section{Solution Methodology}

Separate moment equations are used for the seed and nuclei classes of crystals, and they are defined as : 
\begin{align}
\mu^{n}_{i} = \int_{0}^{r_{g}} r^{i}n(r,t) dr \\
\mu^{s}_{i} = \int_{r_{g}}^{\infty} r^{i}n(r,t) dr
\end{align}
n : nucleated crystal , s: seeded crystal , $r_{g}$  : critical radius separating the two \\  
Since, we ignore the agglomeration and breakage phenomena, the number of seeds added to the process ($\mu_{0}^{s}$) remain constant.\\
Fourth and higher order moments are not affected by the lower, which makes it possible for the complete process dynamics to be expressed by the first 4 moments for each of the respective crystals growth patterns.\\
The moment equations for nucleated and seeded crystals become as follows\cite{yenkie} :

\begin{enumerate}

\item Nucleated crystals\cite{hu}\cite{paeng} 
\begin{align}
\frac{d\mu_{0}^{n}}{dt} &= B(t) \\
\frac{d\mu_{i}^{n}}{dt} &= iG(t)u_{i-1}^{n}(t) \quad  i = 1,2,3
\end{align}

\item Seeded crystals\cite{hu}\cite{paeng}
\begin{align}
\frac{d\mu_{i}^{s}}{dt} &= iG(t)u_{i-1}^{n}(t) \quad  i = 1,2,3 \\
\mu_{0}^{s} &= constant
\end{align}

\end{enumerate}
The total moment is obtained as the summation $\mu_{i}^{t} = \mu_{i}^{n} + \mu_{i}^{s}$. 
The complete set of differential equations are as follows\cite{yenkie} :
\begin{align} 
\frac{dy_{1}}{dt} &= -3\rho k_{v}G(t)(y_{4}+y{8}) \\
\frac{dy_{2}}{dt} &= 0 \\
\frac{dy_{3}}{dt} &= G(t)y_{2}  \\
\frac{dy_{4}}{dt} &= 2G(t)y_{3} \\
\frac{dy_{5}}{dt} &= 3G(t)y_{4} \\
\frac{dy_{6}}{dt} &= B(t)  \\
\frac{dy_{7}}{dt} &= G(t)y_{6}  \\
\frac{dy_{8}}{dt} &= 2G(t)y_{7}  \\
\frac{dy_{9}}{dt} &= 3G(t)y_{8}  \\
\end{align} 
Here the state variables $y_{i}$ are given by : 
\begin{equation*}
y_{i} = \left[\quad C \quad \mu_{0}^{s} \quad \mu_{1}^{s}\quad \mu_{2}^{s}\quad \mu_{3}^{s}\quad \mu_{0}^{n}\quad \mu_{1}^{n}\quad \mu_{2}^{n}\quad \mu_{3}^{n}\quad\right]  
\end{equation*} % Model for the problem

%\input{Chapters/Chapter4} % Experiment 1

%\input{Chapters/Chapter5} % Experiment 2

%\input{Chapters/Chapter6} % Results and Discussion

%\input{Chapters/Chapter7} % Conclusion

%% ----------------------------------------------------------------
% Now begin the Appendices, including them as separate files

\addtocontents{toc}{\vspace{2em}} % Add a gap in the Contents, for aesthetics

\appendix % Cue to tell LaTeX that the following 'chapters' are Appendices

\chapter{An Appendix}

Lorem ipsum dolor sit amet, consectetur adipiscing elit. Vivamus at pulvinar nisi. Phasellus hendrerit, diam placerat interdum iaculis, mauris justo cursus risus, in viverra purus eros at ligula. Ut metus justo, consequat a tristique posuere, laoreet nec nibh. Etiam et scelerisque mauris. Phasellus vel massa magna. Ut non neque id tortor pharetra bibendum vitae sit amet nisi. Duis nec quam quam, sed euismod justo. Pellentesque eu tellus vitae ante tempus malesuada. Nunc accumsan, quam in congue consequat, lectus lectus dapibus erat, id aliquet urna neque at massa. Nulla facilisi. Morbi ullamcorper eleifend posuere. Donec libero leo, faucibus nec bibendum at, mattis et urna. Proin consectetur, nunc ut imperdiet lobortis, magna neque tincidunt lectus, id iaculis nisi justo id nibh. Pellentesque vel sem in erat vulputate faucibus molestie ut lorem.

Quisque tristique urna in lorem laoreet at laoreet quam congue. Donec dolor turpis, blandit non imperdiet aliquet, blandit et felis. In lorem nisi, pretium sit amet vestibulum sed, tempus et sem. Proin non ante turpis. Nulla imperdiet fringilla convallis. Vivamus vel bibendum nisl. Pellentesque justo lectus, molestie vel luctus sed, lobortis in libero. Nulla facilisi. Aliquam erat volutpat. Suspendisse vitae nunc nunc. Sed aliquet est suscipit sapien rhoncus non adipiscing nibh consequat. Aliquam metus urna, faucibus eu vulputate non, luctus eu justo.

Donec urna leo, vulputate vitae porta eu, vehicula blandit libero. Phasellus eget massa et leo condimentum mollis. Nullam molestie, justo at pellentesque vulputate, sapien velit ornare diam, nec gravida lacus augue non diam. Integer mattis lacus id libero ultrices sit amet mollis neque molestie. Integer ut leo eget mi volutpat congue. Vivamus sodales, turpis id venenatis placerat, tellus purus adipiscing magna, eu aliquam nibh dolor id nibh. Pellentesque habitant morbi tristique senectus et netus et malesuada fames ac turpis egestas. Sed cursus convallis quam nec vehicula. Sed vulputate neque eget odio fringilla ac sodales urna feugiat.

Phasellus nisi quam, volutpat non ullamcorper eget, congue fringilla leo. Cras et erat et nibh placerat commodo id ornare est. Nulla facilisi. Aenean pulvinar scelerisque eros eget interdum. Nunc pulvinar magna ut felis varius in hendrerit dolor accumsan. Nunc pellentesque magna quis magna bibendum non laoreet erat tincidunt. Nulla facilisi.

Duis eget massa sem, gravida interdum ipsum. Nulla nunc nisl, hendrerit sit amet commodo vel, varius id tellus. Lorem ipsum dolor sit amet, consectetur adipiscing elit. Nunc ac dolor est. Suspendisse ultrices tincidunt metus eget accumsan. Nullam facilisis, justo vitae convallis sollicitudin, eros augue malesuada metus, nec sagittis diam nibh ut sapien. Duis blandit lectus vitae lorem aliquam nec euismod nisi volutpat. Vestibulum ornare dictum tortor, at faucibus justo tempor non. Nulla facilisi. Cras non massa nunc, eget euismod purus. Nunc metus ipsum, euismod a consectetur vel, hendrerit nec nunc.	% Appendix Title

%\input{Appendices/AppendixB} % Appendix Title

%\input{Appendices/AppendixC} % Appendix Title

\addtocontents{toc}{\vspace{2em}}  % Add a gap in the Contents, for aesthetics
\backmatter

%%

\begin{thebibliography}{9}
\bibitem{mullin}
Mulin, J. W.; Nyvlt, J. Programmed cooling of batch crystallizers.
\textit{Chem. Eng. Sci.} \textbf{1971}, \textit{26}, 369−377.

\bibitem{agjones}
Jones, A. G. Optimal operation of a batch cooling crystallizer.
\textit{Chem. Eng. Sci.} \textbf{1974}, \textit{29}, 1075−1087.
\bibitem{rawlings}
 Rawlings, J. B.; Witkowski, W. R.; Eaton, J. W. Modeling and
control of crystallizers.\textit{ Powder Technol.} \textbf{1992}, \textit{69}, 3−9.
\bibitem{miller_rawlings}
 Miller, S. M.; Rawlings, J. B. Model identification and control
strategies for batch cooling crystallizers. \textit{AIChE J.} \textbf{1994}, \textit{40}, 1312−
1327.
\bibitem{hu}
Hu, Q.; Rohani, S.; Jutan, A. Modelling and optimization of
seeded batch crystallizers. \textit{Comput. Chem. Eng.} \textbf{2005}, \textit{29}, 911−918.

\bibitem{shi}
Shi, D.; El-Farra, N.; Li, M.; Mhaskar, P.; Christofides, P. D.
Predictive control of particle size distribution in particulate processes.
\textit{Chem. Eng. Sci.} \textbf{2006}, \textit{61}, 268−28.

\bibitem{paeng}
Paengjuntuek, W.; Arpornwichanop, A.; Kittisupakorn, P.
Product quality improvement of batch crystallizers by a batch to
batch optimization and non-linear control approach. \textit{Chem. Eng. J.}
\textbf{2008}, \textit{139}, 344−350.
\end{thebibliography}
 ----------------------------------------------------------------
\label{Bibliography}
\lhead{\emph{Bibliography}}  % Change the left side page header to "Bibliography"
\bibliographystyle{unsrtnat}  % Use the "unsrtnat" BibTeX style for formatting the Bibliography
\bibliography{Bibliography}  % The references (bibliography) information are stored in the file named "Bibliography.bib"

\end{document}  % The End
%% ----------------------------------------------------------------