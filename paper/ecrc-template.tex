
% Template for Elsevier CRC journal article
% version 1.1 dated 16 March 2010

% This file (c) 2010 Elsevier Ltd.  Modifications may be freely made,
% provided the edited file is saved under a different name

% This file contains modifications for Procedia Computer Science
% but may easily be adapted to other journals

% Changes since version 1.0
% - elsarticle class option changed from 1p to 3p (to better reflect CRC layout)

%-----------------------------------------------------------------------------------

%% This template uses the elsarticle.cls document class and the extension package ecrc.sty
%% For full documentation on usage of elsarticle.cls, consult the documentation "elsdoc.pdf"
%% Further resources available at http://www.elsevier.com/latex

%-----------------------------------------------------------------------------------

%%%%%%%%%%%%%%%%%%%%%%%%%%%%%%%%%%%%%%%%%%%%%%
%%%%%%%%%%%%%%%%%%%%%%%%%%%%%%%%%%%%%%%%%%%%%%
%%                                          %%
%% Important note on usage                  %%
%% -----------------------                  %%
%% This file must be compiled with PDFLaTeX %%
%% Using standard LaTeX will not work!      %%
%%                                          %%
%%%%%%%%%%%%%%%%%%%%%%%%%%%%%%%%%%%%%%%%%%%%%%
%%%%%%%%%%%%%%%%%%%%%%%%%%%%%%%%%%%%%%%%%%%%%%

%% The '3p' and 'times' class options of elsarticle are used for Elsevier CRC
\documentclass[3p,times]{elsarticle}

%% The `ecrc' package must be called to make the CRC functionality available
\usepackage{ecrc}
\usepackage{amsmath}
%% The ecrc package defines commands needed for running heads and logos.
%% For running heads, you can set the journal name, the volume, the starting page and the authors

%% set the volume if you know. Otherwise `00'
\volume{00}

%% set the starting page if not 1
\firstpage{1}

%% Give the name of the journal
\journalname{Procedia Computer Science}

%% Give the author list to appear in the running head
%% Example \runauth{C.V. Radhakrishnan et al.}
\runauth{}

%% The choice of journal logo is determined by the \jid and \jnltitlelogo commands.
%% A user-supplied logo with the name <\jid>logo.pdf will be inserted if present.
%% e.g. if \jid{yspmi} the system will look for a file yspmilogo.pdf
%% Otherwise the content of \jnltitlelogo will be set between horizontal lines as a default logo

%% Give the abbreviation of the Journal.
\jid{procs}

%% Give a short journal name for the dummy logo (if needed)
\jnltitlelogo{Procedia Computer Science}

%% Hereafter the template follows `elsarticle'.
%% For more details see the existing template files elsarticle-template-harv.tex and elsarticle-template-num.tex.

%% Elsevier CRC generally uses a numbered reference style
%% For this, the conventions of elsarticle-template-num.tex should be followed (included below)
%% If using BibTeX, use the style file elsarticle-num.bst

%% End of ecrc-specific commands
%%%%%%%%%%%%%%%%%%%%%%%%%%%%%%%%%%%%%%%%%%%%%%%%%%%%%%%%%%%%%%%%%%%%%%%%%%

%% The amssymb package provides various useful mathematical symbols
\usepackage{amssymb}
%% The amsthm package provides extended theorem environments
%% \usepackage{amsthm}

%% The lineno packages adds line numbers. Start line numbering with
%% \begin{linenumbers}, end it with \end{linenumbers}. Or switch it on
%% for the whole article with \linenumbers after \end{frontmatter}.
%% \usepackage{lineno}

%% natbib.sty is loaded by default. However, natbib options can be
%% provided with \biboptions{...} command. Following options are
%% valid:

%%   round  -  round parentheses are used (default)
%%   square -  square brackets are used   [option]
%%   curly  -  curly braces are used      {option}
%%   angle  -  angle brackets are used    <option>
%%   semicolon  -  multiple citations separated by semi-colon
%%   colon  - same as semicolon, an earlier confusion
%%   comma  -  separated by comma
%%   numbers-  selects numerical citations
%%   super  -  numerical citations as superscripts
%%   sort   -  sorts multiple citations according to order in ref. list
%%   sort&compress   -  like sort, but also compresses numerical citations
%%   compress - compresses without sorting
%%
%% \biboptions{comma,round}

% \biboptions{}

% if you have landscape tables
\usepackage[figuresright]{rotating}

% put your own definitions here:
%   \newcommand{\cZ}{\cal{Z}}
%   \newtheorem{def}{Definition}[section]
%   ...

% add words to TeX's hyphenation exception list
%\hyphenation{author another created financial paper re-commend-ed Post-Script}

% declarations for front matter

\begin{document}

\begin{frontmatter}

%% Title, authors and addresses

%% use the tnoteref command within \title for footnotes;
%% use the tnotetext command for the associated footnote;
%% use the fnref command within \author or \address for footnotes;
%% use the fntext command for the associated footnote;
%% use the corref command within \author for corresponding author footnotes;
%% use the cortext command for the associated footnote;
%% use the ead command for the email address,
%% and the form \ead[url] for the home page:
%%
%% \title{Title\tnoteref{label1}}
%% \tnotetext[label1]{}
%% \author{Name\corref{cor1}\fnref{label2}}
%% \ead{email address}
%% \ead[url]{home page}
%% \fntext[label2]{}
%% \cortext[cor1]{}
%% \address{Address\fnref{label3}}
%% \fntext[label3]{}

\dochead{}
%% Use \dochead if there is an article header, e.g. \dochead{Short communication}

\title{Deterministic and Stochastic Optimal Control for Batch Cooling Crystallization}

%% use optional labels to link authors explicitly to addresses:
%% \author[label1,label2]{<author name>}
%% \address[label1]{<address>}
%% \address[label2]{<address>}

\author{Tushar Gupta}

\address{}

\begin{abstract}
Minimization of operation costs and the enhancement in product quality have been
major concerns for all industrial processes. The field under study here is batch crystalllization which is affected heavily by the uncertainities in measurements and other
errors.
This work analyses various optimization approaches for the batch crystallization process that are robust to model error. All of the methods involve maximising an objective function by manipulating the cooling profile. At first, the Deterministic approach uses experimental kinetic parameters, which is then extended to Stochastic optimization to incorporate uncertainities in them. Lastly, a novel approach named Polynomial Chaos Expansions is implemented which has been applied successfully to other domains for
Nonlinear Model Predictive Control but was not explored in detail in the field of batch
cooled crystalllization. It successfuly includes probability distributions for the parameters into the model to provide a more robust optimization srategy.

 %The key challenge in addressing robustness to model
%error is to propagate the uncertainty in model parameters onto the
%%control or optimization objective.
\end{abstract}

\begin{keyword}
Stochastic Optimal Control \sep Polynomial Chaos Expansions \sep Robust Optimization \sep Batch Crystallization \sep Predictive Control
\sep Optimum Temperature Profile
%% keywords here, in the form: keyword \sep keyword

%% MSC codes here, in the form: \MSC code \sep code
%% or \MSC[2008] code \sep code (2000 is the default)

\end{keyword}

\end{frontmatter}

%%
%% Start line numbering here if you want
%%
% \linenumbers

%% main text
\section{Introduction}
\label{intro}
Numerous industries today, such as pharmaceutical, chemical, photographic etc. employ the batch crystallization process for the preparation of crystalline products with high degree of purity. A common goal of each crystallization process is to obtain a narrower Particle size distribution (PSD) of the desired product. The PSD has a strong influence on the downstream processing and, hence, reproducible PSD in each operation is of prime importance. Thus, finding an effective control strategy to obtain the resulting crystals with a desired Crystal Size Distribution becomes significant in order for improving the performance of both the batch crystallization process and the subsequent processes which depend on it. \par
Crystallization is the (natural or artificial) process where the atoms or molecules are highly organized into a solid structure known as a crystal. Some of the ways which crystals form are through precipitating 
from a solution, melt or more rarely deposited directly from a gas. In order for crystallization to take place a solution must be "\textit{supersaturated}". \textbf{Supersaturation}$(\Delta{C})$  is a condition in which the solute concentration in the solution is
higher than the solubility. It acts as the driving force for the crystallization process and affects the final quantity of product formed. It is mathematically expressed  as : 
\begin{equation}
\Delta{C} = C - C_{s}
\end{equation}
where $C_{s}$ is the concentration of the solute in the saturated solution.
In the following work, the method in focus is cooling crystallization in which superstauration magnitude is determined by the cooling rate. Thus, determination of an optimal cooling rate or a temperature trajectory becomes the objective of the currrent study. \par 
This work formulates and analyses various control strategies for a cooling crystallization process represented through the population balance equation. \textbf{Deterministic Optimal Control} aims at finding the an optimum temperature profile to maximise an objective function selected to achieve a desired volume of the product.
Herein, the experimental kinetic parameters are employed to simulate a batch crystalllization process. \textbf{Stochastic Optimal Control} undertakes the task of quantifying the uncertainites which creep in due to experimentation. It aims to achive a maximum expected value for the desired product, simultaneously incorporating randomness in the process parameters into the model. Namely, two methods \textbf{Ito Process} and a novel approach \textbf{Polynomial Chaos Expansions} are employed for this purpose. \par

Most of the reported works in the this field deal with the determination of optimal temperature or supersaturation trajectory for the batch crystallizer. The concept of programmed cooling in batch crystallizers was first discussed by Mullin and Nyvlt \cite{mullin} in 1971.
Later, in 1974, A. G. Jones \cite{agjones} presented a mathematical theory based on moment transformations of population balance equations. He used the continuous maximum principle to predict optimal cooling curves.
Rawlings et al. \cite{rawlings} discussed issues in crystal size measurement using laser light scattering experiments and optimal control problem formulation. In 1994, Miller and Rawlings \cite{miller_rawlings}  discussed the uncertain bounds on model parameter estimates for a batch crystallization system. 
ost importantly optimal temperature prediction for batch crystallization has also been done by Hu et al.\lcite{hu}, Shi et al.\cite{shi}, Paengjuntuek et al.\cite{paeng}, and Corriou and Rohani.\cite{corriou}, the data and knowledge from which have been used in further work in this project.\par
Stochastic modeling of particulate processes and parameter estimation using the experimentally measured particle sizes has attracted many researchers. Grosso et al.\cite{grosso} presented a stochastic approach for modeling PSD and comparative assessments of different models. Ma et al.\cite{ma} presented a worse-case performance analysis of optimal control trajectories by considering features such as the computational effort, parametric uncertainty and control implementation inaccuracies. Monte Carlo simulations have also been used to propogate uncertainities but often present the problem of high computational demand, for which approximations are proposed.  Nagy and Braatz (2007) \cite{nagy} have shown that Polynomial Chaos Expansions(PCE) is a computationally efficient alternative to Monte Carlo simulations for propagating uncertainty in dynamic models. PCE is based on orthogonal basis functions thus requiring smaller function evaluations for the calculation of numerical integrations needed for obtaining statistical moments. The computational advantages of PCEs for robust control and optimization has been shown by Nagy and Braatz \cite{nagy}, Kim et al.\cite{kim}, Kumar and Budman\cite{kumar}. \par
The focus of the current research activity is to incorporate parametric uncertainties in the mathematical formulations of batch crystallization process for building a robust model. In the deterministic approach, kinetic parameters from the experimental data have been used to model the system. Next, stochastic Ito processes are used to assimilate the errors in the experimental data. Finally, PCE are demostrated as an effective and novel method to achieve the desired objective function value. A case study of an unseeded crystallization process is also included to authenticate the methodology.

\section{Mathematical Background} \label{model}
This section introduces the underlying concepts needed to build a computational model of a crystallization process which involves regulating the population of particles, also termed as a particulate process. 
The population is described by the density of a suitable extensive variable, usually the \textbf{number of particles}, but sometimes by other variables such as the mass or volume of particles. The usual transport equations expressing conservation laws for material systems apply to the behavior of single particles. Particulate processes are characterized by properties such as particle shape, size, surface area, mass, and product purity. \\
A population balance formulation describes the process of crystal size distribution with time most effectively. Thus, modeling of a batch crystallizer involves the use of population balances to model the crystal size prediction and the mass balance on the system can be modeled as a simple differential equation having concentration as the state variable.
The population balance can be expressed as eq :

\begin{equation} \label{populationbalance}
	\frac{\partial{n(r,t)}}{\partial{t}} + \frac{\partial{G(r,t)n(r,t)}}{\partial{r}} = B  \nolinebreak
\end{equation}
where \textbf{n} is the number density distribution, \textbf{t} is the time, \textbf{r} represents the characteristic dimension for size measurements, \textbf{G} is the crystal growth rate, and \textbf{B} is the nucleation rate. Both growth and nucleation processes describe crystallization kinetics, and their expression may vary, depending on the system under consideration.

In this work, the system under consideration is potassium sulfate, which has been studied earlier for its kinetics by Hu et al. \cite{hu}, Shi et al. \cite{shi}, and Paengjuntuek et al.\cite{paeng}. The expressions are given as follows: \\

Nucleation kinetics$^{(5-7)}$ are defined by :
\begin{equation}
B(t) = k_{b}\exp{\left(-E_{b}/RT \right)}\left(\frac{C - C_{s}(T)}{C_{s}(T)}\right)^{b}\mu_{3}
\end{equation}  


Growth Kinetics$^{(5-7)}$ are given by:
\begin{equation}
G(t) = k_{g}\exp{\left(-E_{g}/RT \right)}\left(\frac{C - C_{s}(T)}{C_{s}(T)}\right)^{g}
\end{equation}
where k$_{b}$ and k$_{g}$ are constants of the system, E$_{b}$ and E$_{g}$ are activation energies, and b and g are exponents of nucleation and growth, respectively. $C_{s}(T)$ is the saturation concentration at a given temperature. The following equations are used to evaluate the saturation and metastable concentrations corresponding to the solution temperature T (expressed in units of $^\circ$C)\cite{shi}. \\
\begin{align}
C_{s}(T) &= 6.29\times10^{-2} + 2.46\times10^{-3}T - 7.14\times10^{-6}T^{2} \\
C_{m}(T) &= 7.76\times10^{-2} + 2.46\times10^{-3}T - 8.1\times10^{-6}T^{2} \label{meta}
\end{align} 
The mass balance, in terms of concentration of the solute in the solution, is expressed as :
\begin{equation}
\frac{dC}{dt} = -3\rho{}k_{v}G(t)\mu_{2}(t)
\end{equation}
where $\rho{}$ is the density of the crystals, $k_{v}$ the volumetric shape factor, and $\mu_{2}$ is the second moment of particle size distribution (PSD).
Since $n(r,t)$ represents the population density of the crystals, the i-th moment of the particle size distribution(PSD) is given by :
\begin{equation} \label{moments}
\mu_{i} = \int_{0}^{\infty} r^{i}n(r,t) dr
\end{equation}
The above equations along with the Population Balance Equation(PBE) represent a complete model of a seeded batch crystallizer. 
Population balance equations are multidimensional, which poses a problem with their implementation in complex control functions, hence use of a model order reduction becomes imperative.\par
For simplification, we reduce the population balance equations into \textbf{Moment balance equations} which has been estabilished as an efficient method by Yenkie et al.\cite{yenkie}. This is done by multiplying the equation (\ref{populationbalance})  with $r^{i}$ on both sides to generate the expression given by equation (\ref{moments}). Converting the model into Ordinary differential equations proves to be advantageous, since it is difficult and time-consuming to formulate an optimization problem involving PBEs. Thus, the moment method leads to a reduced-order model given by Equations (15-24).\\ 
%%%% Insert table of parameters
Separate moment equations are used for the seed and nuclei classes of crystals, and they are defined as : \\
\begin{align}
\mu^{n}_{i} &= \int_{0}^{r_{g}} r^{i}n(r,t) dr \\
\mu^{s}_{i} &= \int_{r_{g}}^{\infty} r^{i}n(r,t) dr
\end{align} \\
\textbf{n} in the superscript represents the nucleated crystal whereas \textbf{s} stands for the seeded crystal, $\boldsymbol{r_{g}}$ gives the critical radius separating the two. The moment equations for nucleated and seeded crystals become as follows\cite{yenkie} :

\begin{enumerate}

\item Nucleated crystals\cite{hu,paeng} 
\begin{align}
\frac{d\mu_{0}^{n}}{dt} &= B(t) \\
\frac{d\mu_{i}^{n}}{dt} &= iG(t)u_{i-1}^{n}(t) \quad  i = 1,2,3
\end{align}

\item Seeded crystals\cite{hu,paeng}
\begin{align}
\mu_{0}^{s} &= constant \\ \label{seed}
\frac{d\mu_{i}^{s}}{dt} &= iG(t)u_{i-1}^{n}(t) \quad  i = 1,2,3 
\end{align}
\end{enumerate}
The total moment is obtained as the summation $\mu_{i}^{t} = \mu_{i}^{n} + \mu_{i}^{s}$. 



\section{Deterministic Optimal Control}

\subsection{\textit{Case Study : Seeded batch crystallization}}
Optimal control involves the evaluation of time-dependent operating profiles, in terms of the control variable to optimize the process performance. In the crystallization domain, temperature becomes the control variable while the product yield is used to evaluate the performance of the model. The Optimization problem here is solved by using the method of Maximum Principle as discussed extensively by Diwekar\cite{dewikar}. This method stands out in compared to other techniques such as dynamic programming as it involves the use of first order ODEs while the later utilises partial differential equations. Another advantage of maximum principle being its ability to extend to stochastic calculus which is explained in the next section. \par 
In a seeded crystallization process it is essential to keep the nucleation phenomena to minimum as in the early stages of growth nucleated crystals might compete with the seeded ones for growth. This ensures uniformity in the shape and size of the final product. It is achieved by incorporating the volume of nucleated in the \textbf{objective fuction}. Third moment($\mu_{3}$) represents volume in a crystallization model as evident from Equation(\ref{moments}) and is used as follows :
\begin{equation}
\max_{T(t)}\lbrace{\mu_{3}^{s}(t_{f}) - \mu_{3}^{n}(t_{f})\rbrace } 
\end{equation}
The active constraints for the process are given by : 
\begin{equation}
C_{s}\leqslant C \leqslant C_{m}
\end{equation}
$C_{m}$ is the metastable concentration described by Equation(\ref{meta}). The state variables $y_{i}$ for the process are given by : 
\begin{equation*}
y_{i} = \left[\quad C \quad \mu_{0}^{s} \quad \mu_{1}^{s}\quad \mu_{2}^{s}\quad \mu_{3}^{s}\quad \mu_{0}^{n}\quad \mu_{1}^{n}\quad \mu_{2}^{n}\quad \mu_{3}^{n}\quad\right]  
\end{equation*}
Using these variable the state equations become \cite{yenkie} :
\begin{align} 
\frac{dy_{1}}{dt} &= -3\rho k_{v}G(t)(y_{4}+y_{8}) \\
\frac{dy_{2}}{dt} &= 0 \\
\frac{dy_{3}}{dt} &= G(t)y_{2}  \\
\frac{dy_{4}}{dt} &= 2G(t)y_{3} \\
\frac{dy_{5}}{dt} &= 3G(t)y_{4} \\
\frac{dy_{6}}{dt} &= B(t)  \\
\frac{dy_{7}}{dt} &= G(t)y_{6}  \\
\frac{dy_{8}}{dt} &= 2G(t)y_{7}  \\
\frac{dy_{9}}{dt} &= 3G(t)y_{8}  \\
\end{align} 


Thus, the complete model involving the moment equations consists of nine state equations.
\subsection{\textit{Solution technique : Steepest ascent Hamiltonian}} 

The algorithm of Steepest Ascent utilizes this principle using the Hamiltonian Derivative to move towards the optimum value of temperature and maximise the objective function. A two-point boundary-value-problem is constructed using state equations and additionally defined adjoint equations which has been explained in detail futher.\\
The formulation results in two point boundary value problem, since initial conditions for the state variables and final conditions for the adjoint variables are available. The method also involves introduction of nine additional variables, known as adjoints ($z_{i}$), corresponding to each of the state variable ($y_{i}$), which must satisfy the \textbf{Hamiltonian equation} represented by :
\begin{equation}
H &= \sum_{i = 1}^{9} z_{i}f(y_{i},t,T) 
\end{equation}
The state variables and objective function can be described as :
\begin{align*}
&\max_{T(t)} \lbrace{ y_{5}(t_{f}) - y_{9}(t_{f})}\rbrace \\
&\frac{dy_{i}}{dt} = f(y_{i},t,T) \\
&\frac{dz_{i}}{dt} = \sum_{j=1}^{9} z_{j}\frac{\partial f(y_{i},t,T)}{\partial y_{i}} = f(y_{i},z_{i},t,T) \\
\end{align*}
with the following initial conditions:\\
$t_{0} = 0$ and $t_{f} = 1800s$ (batch time)
\begin{align*}
&y_{i}(t_{0}) = \left[ 0.1743 \quad 66.66 \quad 1.83\times10^{4}\quad 5.05\times10^{6} \quad 1.93\times10^{9} \quad 0.867 \quad 0 \quad 0 \quad 0 \right] \\
&z_{i}(t_{f}) = \left[  0 \quad 0 \quad 0 \quad 0 \quad 1 \quad 0 \quad 0 \quad 0 \quad -1 \right] 
\end{align*}
\paragraph{Algorithm}
\begin{enumerate}
\item An initial temperature $T(t) = 323 K$ is assumed for the entire time horizon.
\item The differential equations for state variables are integrated using the initial conditions for a time step of 1s .
\item The value of the adjoint variables are computed by backward integration for the same time step used in the previous.
\item For evaluation of the Hamiltonian derivative, an analytical method proposed by Benavides and Diwekar\cite{benavides},  is used in an additional variable corresponding to each of the state and adjoint variable is introduced.
\item The variable $\theta_{i}$ corresponds to each of the state variable $y_{i}$ and the variable $\phi_{i}$ corresponds to each of the adjoint variable $z_{i}$, respectively.
\item The Hamiltonian derivative is now calculated at each time step  as :
\begin{align}
&\theta = \frac{dy_{i}}{dT} \quad and \quad \phi_{i} = \frac{dz_{i}}{dT} \\
&\frac{dH}{dT} = \sum_{i=1}^{9} \left( \frac{dH}{dy_{i}}\right)\left(	\frac{dy_{i}}{dT} \right) + \sum_{i=1}^{9} \left(\frac{dH}{dz_{i}}\right)\left(\frac{dz_{i}}{dT} \right)
\end{align}
\item The  convergence criterion $(\frac{dH}{dT}<$ tolerance) is verified. If it is not satisfied, the temperature $T(t)$ is updated using this gradient\cite{yenkie}.
\begin{equation}
T^{new}(t) = T^{old}(t) + M\left(\frac{dH}{dT} \right)
\end{equation}
\item The concentration is evaluated  at that time step and compared with first with the saturation concentration ($C_{s}$) and the metastable concentration($C_{m}$) to validate the active constraints.
\item Iterations of above steps are repeated.
\end{enumerate}

%% The Appendices part is started with the command \appendix;
%% appendix sections are then done as normal sections
%% \appendix

%% \section{}
%% \label{}

%% References
%%
%% Following citation commands can be used in the body text:
%% Usage of \cite is as follows:
%%   \cite{key}         ==>>  [#]
%%   \cite[chap. 2]{key} ==>> [#, chap. 2]
%%

%% References with BibTeX database:


%\bibliography{mybibfile}
%\bibliographystyle{ieeetr}
%% Authors are advised to use a BibTeX database file for their reference list.
%% The provided style file elsarticle-num.bst formats references in the required Procedia style

%% For references without a BibTeX database:
\newpage
\section{References}
\begin{thebibliography}{1}

%\bibitem must have the following form:
 \bibitem{key}...
	
 %\bibitem{}

\end{thebibliography}

\end{document}

%%
%% End of file `ecrc-template.tex'. 