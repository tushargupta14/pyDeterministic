
% Template for Elsevier CRC journal article
% version 1.1 dated 16 March 2010

% This file (c) 2010 Elsevier Ltd.  Modifications may be freely made,
% provided the edited file is saved under a different name

% This file contains modifications for Procedia Computer Science
% but may easily be adapted to other journals

% Changes since version 1.0
% - elsarticle class option changed from 1p to 3p (to better reflect CRC layout)

%-----------------------------------------------------------------------------------

%% This template uses the elsarticle.cls document class and the extension package ecrc.sty
%% For full documentation on usage of elsarticle.cls, consult the documentation "elsdoc.pdf"
%% Further resources available at http://www.elsevier.com/latex

%-----------------------------------------------------------------------------------

%%%%%%%%%%%%%%%%%%%%%%%%%%%%%%%%%%%%%%%%%%%%%%
%%%%%%%%%%%%%%%%%%%%%%%%%%%%%%%%%%%%%%%%%%%%%%
%%                                          %%
%% Important note on usage                  %%
%% -----------------------                  %%
%% This file must be compiled with PDFLaTeX %%
%% Using standard LaTeX will not work!      %%
%%                                          %%
%%%%%%%%%%%%%%%%%%%%%%%%%%%%%%%%%%%%%%%%%%%%%%
%%%%%%%%%%%%%%%%%%%%%%%%%%%%%%%%%%%%%%%%%%%%%%

%% The '3p' and 'times' class options of elsarticle are used for Elsevier CRC
\documentclass[3p,times]{elsarticle}

%% The `ecrc' package must be called to make the CRC functionality available
\usepackage{ecrc}
\usepackage{biblatex}
\addbibresource{mybibfile.bib}
%% The ecrc package defines commands needed for running heads and logos.
%% For running heads, you can set the journal name, the volume, the starting page and the authors

%% set the volume if you know. Otherwise `00'
\volume{00}

%% set the starting page if not 1
\firstpage{1}

%% Give the name of the journal
\journalname{Procedia Computer Science}

%% Give the author list to appear in the running head
%% Example \runauth{C.V. Radhakrishnan et al.}
\runauth{}

%% The choice of journal logo is determined by the \jid and \jnltitlelogo commands.
%% A user-supplied logo with the name <\jid>logo.pdf will be inserted if present.
%% e.g. if \jid{yspmi} the system will look for a file yspmilogo.pdf
%% Otherwise the content of \jnltitlelogo will be set between horizontal lines as a default logo

%% Give the abbreviation of the Journal.
\jid{procs}

%% Give a short journal name for the dummy logo (if needed)
\jnltitlelogo{Procedia Computer Science}

%% Hereafter the template follows `elsarticle'.
%% For more details see the existing template files elsarticle-template-harv.tex and elsarticle-template-num.tex.

%% Elsevier CRC generally uses a numbered reference style
%% For this, the conventions of elsarticle-template-num.tex should be followed (included below)
%% If using BibTeX, use the style file elsarticle-num.bst

%% End of ecrc-specific commands
%%%%%%%%%%%%%%%%%%%%%%%%%%%%%%%%%%%%%%%%%%%%%%%%%%%%%%%%%%%%%%%%%%%%%%%%%%

%% The amssymb package provides various useful mathematical symbols
\usepackage{amssymb}
%% The amsthm package provides extended theorem environments
%% \usepackage{amsthm}

%% The lineno packages adds line numbers. Start line numbering with
%% \begin{linenumbers}, end it with \end{linenumbers}. Or switch it on
%% for the whole article with \linenumbers after \end{frontmatter}.
%% \usepackage{lineno}

%% natbib.sty is loaded by default. However, natbib options can be
%% provided with \biboptions{...} command. Following options are
%% valid:

%%   round  -  round parentheses are used (default)
%%   square -  square brackets are used   [option]
%%   curly  -  curly braces are used      {option}
%%   angle  -  angle brackets are used    <option>
%%   semicolon  -  multiple citations separated by semi-colon
%%   colon  - same as semicolon, an earlier confusion
%%   comma  -  separated by comma
%%   numbers-  selects numerical citations
%%   super  -  numerical citations as superscripts
%%   sort   -  sorts multiple citations according to order in ref. list
%%   sort&compress   -  like sort, but also compresses numerical citations
%%   compress - compresses without sorting
%%
%% \biboptions{comma,round}

% \biboptions{}

% if you have landscape tables
\usepackage[figuresright]{rotating}

% put your own definitions here:
%   \newcommand{\cZ}{\cal{Z}}
%   \newtheorem{def}{Definition}[section]
%   ...

% add words to TeX's hyphenation exception list
%\hyphenation{author another created financial paper re-commend-ed Post-Script}

% declarations for front matter

\begin{document}

\begin{frontmatter}

%% Title, authors and addresses

%% use the tnoteref command within \title for footnotes;
%% use the tnotetext command for the associated footnote;
%% use the fnref command within \author or \address for footnotes;
%% use the fntext command for the associated footnote;
%% use the corref command within \author for corresponding author footnotes;
%% use the cortext command for the associated footnote;
%% use the ead command for the email address,
%% and the form \ead[url] for the home page:
%%
%% \title{Title\tnoteref{label1}}
%% \tnotetext[label1]{}
%% \author{Name\corref{cor1}\fnref{label2}}
%% \ead{email address}
%% \ead[url]{home page}
%% \fntext[label2]{}
%% \cortext[cor1]{}
%% \address{Address\fnref{label3}}
%% \fntext[label3]{}

\dochead{}
%% Use \dochead if there is an article header, e.g. \dochead{Short communication}

\title{Deterministic and Stochastic Optimal Control for Batch Cooling Crystallization}

%% use optional labels to link authors explicitly to addresses:
%% \author[label1,label2]{<author name>}
%% \address[label1]{<address>}
%% \address[label2]{<address>}

\author{Tushar Gupta}

\address{}

\begin{abstract}
Minimization of operation costs and the enhancement in product quality have been
major concerns for all industrial processes. The field under study here is batch crystalllization which is affected heavily by the uncertainities in measurements and other
errors.
The current work aims to study Deterministic and Stochastic methods for Optimum
Control in batch crystallization. All the methods involve maximising an objective function by manipulating the cooling profile. At first, the Deterministic approach uses
experimental kinetic parameters, which is then extended to Stochastic optimization to
incorporate uncertainities in them. Lastly, a novel approach named Polynomial Chaos
Expansions is implemented which has been applied successfully to other domains for
Nonlinear Model Predictive Control but was not explored in detail in the field of batch
cooled crystalllization. It successfuly includes probability distributions for the parameters into the model to provide a more robust optimization srategy.

This work analyses various optimization approaches for the batch crystallization process that are robust to model error. First, The key challenge in addressing robustness to model
error is to propagate the uncertainty in model parameters onto the
control or optimization objective.
\end{abstract}

\begin{keyword}
Stochastic Optimal Control \sep Polynomial Chaos Expansions \sep Robust Optimization \sep Batch Crystallization \sep Predictive Control
\sep Optimum Temperature Profile
%% keywords here, in the form: keyword \sep keyword

%% MSC codes here, in the form: \MSC code \sep code
%% or \MSC[2008] code \sep code (2000 is the default)

\end{keyword}

\end{frontmatter}

%%
%% Start line numbering here if you want
%%
% \linenumbers

%% main text
\section{Introduction}
\label{intro}
Numerous industries today, such as pharmaceutical, chemical, photographic etc. employ the Batch crystallization process for the preparation of crystalline products with high degree of purity. A common goal of each crystallization process is to obtain a narrower Particle size distribution (PSD) of the desired product. The PSD has a strong influence on the downstream processing and, hence, reproducible PSD in each operation is of prime importance. Thus, finding an effective control strategy to obtain the resulting crystals with a desired Crystal Size Distribution becomes significant in order for improving the performance of both the batch crystallization process and the subsequent processes which depend on it. \par
Crystallization is the (natural or artificial) process where the atoms
or molecules are highly organized into a solid structure known as a crystal. Some of the ways which crystals form are through precipitating 
from a solution, melt or more rarely deposited directly from a gas. In order for crystallization to take place a solution must be "\textit{supersaturated}". \textbf{Supersaturation}$(\Delta{C})$  is a condition in which the solute concentration in the solution is
higher than the solubility. It acts as the driving force for the crystallization process and is mathematically expressed  as : 
\begin{equation}
\Delta{C} = C - C_{s}
\end{equation}
where $C_{s}$ is the concentration of the solute in the saturated solution.
In the following work, the method in focus is cooling crystallization in which superstauration magnitude is determined by the cooling rate. Thus, determination of an optimal cooling rate or a temperature trajectory is the objective of the currrent research work. \par 
This work formulates and analyses various control strategies for a cooling crystallization process described by using the population balance equation and obtain the solution for the optimal temperature profile using deterministic and stochastic methods. \textbf{Deterministic Optimal Control} aims at finding the an optimum temperature profile to maximise an objective function selected to achieve a desired volume of the product.
Herein, the experimental kinetic parameters are employed to simulate a batch crystalllization process. \textbf{Stochastic Optimal Control} undertakes the task of quantifying the uncertainites which creep in due to experimentation. It aims to achive a maximum expected value for the desired product, simultaneously incorporating randomness in the process parameters into the model. Namely, Two methods \textbf{Ito Process} and a novel approach \textbf{Polynomial Chaos Expansions} are employed for this purpose. \par

Most of the reported works in the this field deal with the determination of optimal temperature or supersaturation trajectory for the batch crystallizer. The concept of programmed cooling in batch crystallizers was first discussed by Mullin and Nyvlt \cite{mullin} in 1971. They studied the laboratory-scale crystallization of potassium sulfate and ammonium sulfate using a temperature controller and observed improvement in the crystal size and quality under programmed cooling. 
Later, in 1974, A. G. Jones \cite{agjones} presented a mathematical theory based on moment transformations of population balance equations. He used the continuous maximum principle to predict optimal cooling curves.
Rawlings et al. \cite{rawlings} discussed issues in crystal size measurement using laser light scattering experiments and optimal control problem formulation. In 1994, Miller and Rawlings \cite{miller_rawlings}  discussed the uncertain bounds on model parameter estimates for a batch crystallization system. 
Most importantly optimal temperature prediction for batch crystallization has also been done by Hu et al.\cite{hu}, Shi et al.\cite{shi}, Paengjuntuek et al.\cite{paeng}, and Corriou and Rohani.\cite{corriou}, the data and knowledge from which have been used in further work in this project . Grosso et al.\cite{grosso} presented a stochastic approach for modeling PSD and comparative assessments of different models. Ma et al.\cite{ma} presented a worse-case performance analysis of optimal control trajectories by considering features such as the computational effort, parametric uncertainty and control implementation inaccuracies. 
The focus of the current work is to be able to handle parametric uncertainties in mathematical formulations of batch crystallization process.
\cite{Dirac1953888}



%% The Appendices part is started with the command \appendix;
%% appendix sections are then done as normal sections
%% \appendix

%% \section{}
%% \label{}

%% References
%%
%% Following citation commands can be used in the body text:
%% Usage of \cite is as follows:
%%   \cite{key}         ==>>  [#]
%%   \cite[chap. 2]{key} ==>> [#, chap. 2]
%%

%% References with BibTeX database:

%\bibliography{mybibfile}
%\bibliographystyle{ieeetr}
%% Authors are advised to use a BibTeX database file for their reference list.
%% The provided style file elsarticle-num.bst formats references in the required Procedia style

%% For references without a BibTeX database:

%\begin{thebibliography}{00}

%\bibitem must have the following form:
 %\bibitem{key}...
	
 %\bibitem{}

%\end{thebibliography}

%\end{document}

%%
%% End of file `ecrc-template.tex'. 